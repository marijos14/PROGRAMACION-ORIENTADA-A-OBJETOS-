\documentclass[letterpaper,12pt,oneside]{article}

\usepackage[top=1in, left=1.25in, right=1.25in, bottom=1in]{geometry}
\usepackage[T1]{fontenc}
\usepackage[utf8]{inputenc}
\usepackage[spanish,es-nodecimaldot,es-tabla]{babel}
\usepackage{caption, subcaption}
\usepackage{graphicx}
\usepackage{array}
\usepackage{tikz}
\usepackage{imakeidx}
\usepackage[style=numeric]{biblatex}
\usepackage{csquotes}
\usepackage{placeins}
\usepackage{float}
\usepackage{setspace}

\addbibresource{./bib/protocolo.bib}
\graphicspath{./figs/}

\begin{document}
\begin{titlepage}
    \centering
    \includegraphics[width=1\textwidth]{Carátula reportes POO.pdf}
\end{titlepage}

\tableofcontents
\clearpage 

\section{Introducción} %Este apartado debe abordar los siguientes puntos:

\begin{itemize}
    \item \textbf{Planteamiento del problema:}\\ %Se hace una descripción del problema a resolver.
    Desarrollar una aplicación que simule el sistema de batallas de Pokémon. La aplicación debe gestionar un combate por turnos entre usuario y rival, determinando el orden de ataque con base en la velocidad y finalizando el encuentro cuando la salud de alguno llegue a cero. Asimismo, el sistema debe implementar la lógica de efectividad de tipos de acuerdo con la tabla de tipos y ofrecer una interfaz gráfica interactiva que permita seleccionar ataques y visualizar el estado actual del combate.
    \item \textbf{Motivación:}\\ %Se describe por qué es necesario dar solución al problema.
    La simulación de un sistema de batalla por turnos representa un escenario ideal para la implementación avanzada de POO, ya que requiere la interacción constante entre múltiples objetos con estados complejos. Utilizar Flutter permite aplicar lógica de backend y comprender cómo los objetos se vinculan con una interfaz gráfica reactiva, usando el patrón de diseño MVC y acercando el desarrollo a un entorno de producción de software moderno y funcional.
    \item \textbf{Objetivos:}\\ %Lo que se espera obtener al darle solución al problema.
    Implementar una jerarquía de clases que modele correctamente las entidades del juego aplicando conceptos fundamentales como herencia, polimorfismo, clases abstractas y el patrón de diseño MVC, así como integrar la lógica con la interfaz de usuario mediante el manejo de estados en Flutter, asegurando que las reglas del juego se ejecuten de manera consistente y transparente para el usuario.
\end{itemize}

\section{Marco Teórico}


\subsection{Dart}
Dart es un lenguaje de programación moderno, desarrollado por Google, que combina el poder de la programación orientada a objetos con la facilidad y eficiencia de los lenguajes de programación basados en scripts [1].

\subsection{Clases y Objetos}
Una clase es una plantilla que define atributos y comportamientos, mientras que un objeto es una instancia concreta de dicha clase. En Dart, las clases permiten representar entidades del sistema como Pokémon o ataques, y facilitan la modularidad del código [2].

\subsection{Herencia y Clases Abstractas}
Son los mecanismos que permiten crear jerarquías donde una clase base define
comportamientos comunes que son extendidos y concretados por subclases específicas.
Esto facilita la reutilización de código y el polimorfismo [3].

\subsection{Polimorfismo}
El polimorfismo permite que un mismo método pueda tener comportamientos diferentes dependiendo del objeto que lo invoque. En el sistema de batalla, ataques de distintos tipos pueden sobrescribir métodos para calcular daño de forma específica.

\subsection{Encapsulamiento}
Consiste en ocultar los detalles internos de una clase y exponer solo lo necesario mediante métodos públicos. Esto permite gestionar de forma segura atributos como puntos de salud (HP) o velocidad, evitando modificaciones no controladas.

\subsection{Constructores y Parámetros Nombrados}
Los constructores permiten inicializar objetos y, con los parámetros nombrados, Dart facilita la legibilidad y claridad al crear entidades complejas como ataques o Pokémon con múltiples atributos.

\subsection{Colecciones y Listas}
Dart proporciona listas dinámicas que permiten almacenar colecciones de objetos, como los movimientos de un Pokémon.

\subsection{Manejo de Estados y Lógica de Negocio}
La lógica de turnos, aplicación de daño, cambios de estado y verificación de condiciones de victoria se implementan mediante métodos dentro de las clases del sistema. El uso de estructuras condicionales y funciones permite controlar cada etapa del combate.

\subsection{Uso de la Biblioteca \texttt{dart:math}}
Se emplea para generar valores aleatorios en ataques, o variaciones en el daño, mediante la clase \texttt{Random}.

\subsection{Flutter}
Flutter es un \textit{framework} basado en widgets, componentes visuales inmutables que describen la estructura de la interfaz. Cada pantalla o elemento gráfico (selección de ataques, barra de vida, menús) se construye con widgets [4].

\textbf{Stateful Widgets.}
Se utilizan cuando la interfaz debe reaccionar a cambios en el estado interno, como la actualización del HP o el avance del combate por turnos.

\textbf{Gestión de Estado.}
El método \texttt{setState()} permite actualizar variables y reflejar cambios en la interfaz. Esto es fundamental para mostrar en tiempo real la vida restante, los mensajes de combate o la disponibilidad de acciones.

\subsection{Mecánicas de Combate Pokémon}

\textbf{HP (Health Points):}
``Es la condición física del Pokémon, representada con un valor
numérico. Estos son reducidos normalmente mediante los ataques del oponente
en combate, los efectos del veneno, las quemaduras, o varios climas entre otros
medios'' [5].

\textbf{Velocidad:}
``La velocidad es la propiedad del Pokémon de atacar, antes o después,
que el oponente. A la hora de atacar el Pokémon con un mayor valor de velocidad,
por lo general, siempre atacará primero''.

\textbf{Efectividad por Tipos.}
Cada tipo de ataque puede ser más o menos efectivo dependiendo del tipo del oponente.
Esta efectividad se expresa como multiplicadores: neutral (1), supereficaz (2), poco eficaz (0.5) o sin efecto (0) [6].
\begin{figure} [H]
    \centering
    \includegraphics[width=0.75\linewidth]{PROYECTO-03//Reporte//Borradores/TablaTipos.png}
    \caption{Tabla de tipos de Pokémon. [6]}
\end{figure}

\subsection{Patrones de Diseño}

\textbf{Modelo–Vista–Controlador (MVC).}
El patrón MVC separa la lógica de negocio (modelo), la interfaz de usuario (vista) y el flujo de datos (controlador). En Flutter, esta separación se logra mediante la distribución de clases y widgets, permitiendo un proyecto organizado, escalable y sostenible [7].

\section{Desarrollo} %Es la descripción de la implementación realizada en el lenguaje de programación, así como las pruebas realizadas para obtener los resultados. No se debe mostrar código, solo describir funciones o la aplicación de los conceptos teóricos. Las pruebas es describir las entradas ingresadas y las salidas obtenidas.

\subsection*{Clase 1}

\subsection*{Clase 2}

\subsection*{Clase 3}

\subsection*{Clase 4}

\subsection*{Diagrama de clases (UML estático)}

\begin{figure} [H]
    \centering
    \includegraphics[width=1\linewidth]{PROYECTO-03/Reporte/Borradores/estatica1.png}
    \caption{UML Estático: Diagrama de clases de Pokemon.}
\end{figure}

\begin{figure} [H]
    \centering
    \includegraphics[width=1\linewidth]{PROYECTO-03/Reporte/Borradores/estatica2.png}
    \caption{UML Estático: Diagrama de clases de Ataque.}
\end{figure}

\begin{figure} [H]
    \centering
    \includegraphics[width=1\linewidth]{PROYECTO-03/Reporte/Borradores/estatica3.png}
    \caption{UML Estático: Diagrama de clases de main.}
\end{figure}
   
\subsection*{Diagrama de secuencia (UML dinámico)}
\begin{figure}[H]
    \centering
    \includegraphics[width=1\linewidth]{PROYECTO-03/Reporte/Borradores/UML dinamico_ diagrama de secuencia_main.png}
    \caption{UML Dinámico: Diagrama de secuencia del main.}
\end{figure}

\begin{figure}[H]
    \centering
    \includegraphics[width=1\linewidth]{PROYECTO-03/Reporte/Borradores/UML dinamico_ diagrama de secuencia_combate_view.png}
    \caption{UML Dinámico: Diagrama de secuencia de combate view.}
\end{figure}

\begin{figure}[H]
    \centering
    \includegraphics[width=0.8\linewidth]{PROYECTO-03/Reporte/Borradores/UML dinamico_ diagrama de secuencia_combate_controller.png}
    \caption{UML Dinámico: Diagrama de secuencia de combate controller.}
\end{figure}

\begin{figure}[H]
    \centering
    \includegraphics[width=1\linewidth]{PROYECTO-03/Reporte/Borradores/UML dinamico_ diagrama de secuencia_seleccion_page.png}
    \caption{UML Dinámico: Diagrama de secuencia de seleccion page.}
\end{figure}

\begin{figure}[H]
    \centering
    \includegraphics[width=1\linewidth]{PROYECTO-03/Reporte/Borradores/UML dinamico_ diagrama de secuencia_batalla_page.png}
    \caption{UML Dinámico: Diagrama de secuencia de batalla page.}
\end{figure}

\section{Resultados} %Mediante capturas de pantalla y una breve descripción seguida de la captura se presentan los resultados finales de su aplicación.
\begin{figure} [H]
    \centering
    \includegraphics[width=0.8\linewidth]{PROYECTO-03/Reporte/Borradores/Eleccion.png}
    \caption{Pantalla inicial donde se escoge el Pokémon del usuario y el Pokémon rival.}
\end{figure}
\begin{figure} [H]
    \centering
    \includegraphics[width=0.8\linewidth]{PROYECTO-03/Reporte/Borradores/Batalla1.png}
    \caption{Pantalla de batalla inicial; batalla de exhibición.}
\end{figure}
\begin{figure} [H]
    \centering
    \includegraphics[width=0.8\linewidth]{PROYECTO-03/Reporte/Borradores/Ataques.png}
    \caption{Pantalla de batalla, menú de ataques (mostrando solo los ataques disponibles del Pokémon elegido).}
\end{figure}
\begin{figure} [H]
    \centering
    \includegraphics[width=0.5\linewidth]{PROYECTO-03/Reporte/Borradores/Logbatalla.png}
    \caption{Pantalla de batalla, historial de batalla que muestra los movimientos realizados por ambos pokemones.}
\end{figure}
\begin{figure} [H]
    \centering
    \includegraphics[width=0.8\linewidth]{PROYECTO-03/Reporte/Borradores/Batalla2.png}
    \caption{Pantalla de batalla, fin de la batalla con los resultados correspondientes.}
\end{figure}

\section{Conclusiones}
El desarrollo de este proyecto permitió integrar de manera efectiva el paradigma de la Programación Orientada a Objetos con la lógica de desarrollo de aplicaciones en Flutter. Se comprobó que el uso de clases abstractas y herencia es esencial para gestionar la diversidad de los elementos de un programa sin duplicar código, permitiendo que el sistema sea escalable y sostenible.

La implementación de la tabla de tipos y el cálculo de daño evidenció la importancia de centralizar las reglas en valores finales para evitar inconsistencias en el estado de la batalla. Además, la gestión de turnos basada en la velocidad reforzó la comprensión del flujo de control y la manipulación de objetos en tiempo de ejecución. Finalmente, el uso de Flutter y el patrón de diseño MVC demostró cómo la programación orientada a objetos sirve como base sólida para construir interfaces gráficas dinámicas, donde cada elemento visual responde a los cambios en los modelos de datos.
section*{Referencias}
\begin{thebibliography}{9}

\bibitem{dart}
Google. \textbf{A Tour of the Dart Language}. URL: \url{https://dart.dev/language}

\bibitem{dartClasses}
Dart Language Documentation. \textbf{Classes}. Nov. 2025. URL: \url{https://dart.dev/language/classes}

\bibitem{dartExtend}
Dart Language Documentation. \textbf{Object-Oriented Programming in Dart}. Nov. 2025. URL: \url{https://dart.dev/language/extend}

\bibitem{flutterArch}
Flutter Documentation. \textbf{Architectural Overview}. Dic. 2025. URL: \url{https://docs.flutter.dev/resources/architectural-overview}

\bibitem{pokeStats}
Wikidex. \textbf{Mecánica - Características}. Nov. 2025. URL: \url{https://www.wikidex.net/wiki/Caracter%C3%ADsticas}

\bibitem{pokeTypes}
Wikidex. \textbf{Mecánica - Tipos}. Nov. 2025. URL: \url{https://www.wikidex.net/wiki/Tipo#Efectividades_de_los_tipos}

\bibitem{mvc}
Rhm, F. (2023). \textbf{Understanding MVC Architecture in Flutter: A Comprehensive Guide with Examples}. Medium. URL: \url{https://medium.com/@Faiz_Rhm/understanding-mvc-architecture-in-flutter-a-comprehensive-guide-with-examples-5d1a372c7eaf}
\end{thebibliography}
\end{document}