\documentclass[letterpaper,12pt,oneside]{article}

\usepackage[top=1in, left=1.25in, right=1.25in, bottom=1in]{geometry}
\usepackage[T1]{fontenc}
\usepackage[utf8]{inputenc}
\usepackage[spanish,es-nodecimaldot,es-tabla]{babel}
\usepackage{caption, subcaption}
\usepackage{graphicx}
\usepackage{array}
\usepackage{tikz}
\usepackage{imakeidx}
\usepackage[style=numeric]{biblatex}
\usepackage{csquotes}
\usepackage{placeins}
\usepackage{float}
\usepackage{setspace}

\addbibresource{./bib/protocolo.bib}
\graphicspath{./figs/}

\begin{document}
\begin{titlepage}
    \centering
    \includegraphics[width=1\textwidth]{Carátula reportes POO.pdf}
\end{titlepage}

\tableofcontents
\clearpage

\section{Introducción} %Este apartado debe abordar los siguientes puntos:

\begin{itemize}
    \item \textbf{Planteamiento del problema:}\\ %Se hace una descripción del problema a resolver.
    Analizar el código proporcionado para identificar y explicar la aplicación práctica de los temas de Archivos, Hilos y Patrones de Diseño en los diferentes ejemplos. Además de documentar la arquitectura estática y dinámica de los programas mediante Diagramas UML y dar la interpretación de los conceptos teóricos en el contexto del código.
    \item \textbf{Motivación:}\\ %Se describe por qué es necesario dar solución al problema.
    La combinación de la persistencia de datos (Archivos), la gestión de tareas concurrentes (Hilos) y las soluciones arquitectónicas comprobadas (Patrones de Diseño) es fundamental en el desarrollo de software complejo. Esta práctica permite integrar conocimientos avanzados de POO, fortaleciendo la capacidad de construir aplicaciones robustas, eficientes y sostenibles, que pueden gestionar recursos y ejecutar tareas de manera óptima.
    \item \textbf{Objetivos:}\\ %Lo que se espera obtener al darle solución al problema.
    Aplicar los conceptos de manejo de archivos para la persistencia de información y ejecutar procesos concurrentes mediante hilos para mejorar la eficiencia del sistema, así como identificar y modelar los patrones de diseño implementados en el código proporcionado. Además, reforzar la habilidad de documentar la arquitectura del software mediante la creación de Diagramas UML estáticos y dinámicos.
\end{itemize}

\section{Marco Teórico}

En esta práctica vamos a aplicar tres conceptos muy importantes para el desarrollo de aplicaciones en Java: el manejo de archivos para guardar datos, el uso de hilos para hacer tareas al mismo tiempo y los patrones de diseño para organizar mejor nuestro código.

\subsection{Archivos (Persistencia)}
Normalmente, cuando cerramos un programa, todas las variables y datos que estaban en la memoria RAM se borran. Para evitar esto y que la información se quede guardada ("persistencia"), usamos archivos.

En Java, esto se maneja principalmente con \textbf{Streams} (flujos de datos):
\begin{itemize}
    \item \textbf{Archivos de Texto:} Son los que podemos leer nosotros, como un \texttt{.txt}. Usamos clases como \texttt{FileReader} o \texttt{BufferedReader} para leer línea por línea.
    \item \textbf{Archivos Binarios:} Guardan la información en bytes (unos y ceros). Son útiles para guardar objetos completos o imágenes.
    \item \textbf{Serialización:} Es el proceso de convertir un objeto de Java en una secuencia de bytes para poder guardarlo en un archivo y recuperarlo después tal cual estaba.
\end{itemize}

\subsection{Hilos }
Un hilo o \textit{thread} es la unidad básica de ejecución de un programa. Usar hilos nos permite hacer \textbf{multitarea}, es decir, que nuestro programa haga dos o más cosas a la vez (por ejemplo, mostrar una interfaz gráfica mientras descarga un archivo de fondo).

Hay dos formas principales de crear hilos en Java:
\begin{enumerate}
    \item Heredando de la clase \texttt{Thread}: Creamos una clase que extienda a Thread y sobrescribimos el método \texttt{run()}.
    \item Implementando la interfaz \texttt{Runnable}: Es la forma más recomendada. Nuestra clase implementa \texttt{Runnable} y definimos qué debe hacer el hilo en el método \texttt{run()}.
\end{enumerate}

Los hilos tienen un ciclo de vida: nacen, se ejecutan, a veces se pausan (esperan) y finalmente mueren cuando terminan su tarea.

\subsection{Patrones de Diseño}
Los patrones de diseño son como "recetas" o soluciones estándar para problemas que ocurren muy seguido al programar. Nos ayudan a que el código sea más ordenado y fácil de modificar. Se dividen en tres tipos:

\begin{itemize}
    \item \textbf{Creacionales:} Nos dicen cómo crear objetos de forma eficiente (Ejemplo: \textit{Singleton}, que asegura que solo exista una instancia de una clase).
    \item \textbf{Estructurales:} Nos ayudan a conectar clases y objetos para formar estructuras más grandes (Ejemplo: \textit{Adapter}).
    \item \textbf{De Comportamiento:} Se encargan de la comunicación entre objetos (Ejemplo: \textit{Observer}, que avisa a otros objetos cuando algo cambia).
\end{itemize}


\section{Desarrollo} %Es la descripción de la implementación realizada en el lenguaje de programación, así como las pruebas realizadas para obtener los resultados. No se debe mostrar código, solo describir funciones o la aplicación de los conceptos teóricos. Las pruebas es describir las entradas ingresadas y las salidas obtenidas.

\subsection*{Clase 1}

\subsection*{Clase 2}

\subsection*{Clase 3}

\subsection*{Clase 4}

\subsection*{Diagrama de clases (UML estático)}
%Se presenta un diagrama de clases que modela la aplicacion de forma estática

\subsection*{Diagrama de secuencia (UML dinámico)}
%Se presenta un diagrama de secuencia que modela la aplicacion de forma dinamica
\vspace{-0.3cm} % ← reduce espacio antes de la figura

\begin{figure}[H]
    \centering
    \includegraphics[width=0.75\textwidth]{Diagrama de secuencia (Dinamico)_Ejemplo 2.png}
    \caption{Diagrama de Secuencia: Ejemplo 2}
    \label{fig:diagrama-secuencia}
\end{figure}

\vspace{0.5cm} % ← espacio después de la figura
\begin{figure}[H]
    \centering
    \includegraphics[width=0.75\textwidth]{Diagrama de secuencia (Dinamico)_Ejemplo 4.png}
    \caption{Diagrama de Secuencia: Ejemplo 4}
    \label{fig:diagrama-secuencia}
\end{figure}

\vspace{0.5cm} % ← espacio después de la figura
\begin{figure}[H]
    \centering
    \includegraphics[width=0.75\textwidth]{Diagrama de secuencia (Dinamico)_Ejemplo 5.png}
    \caption{Diagrama de Secuencia: Ejemplo 5}
    \label{fig:diagrama-secuencia}
\end{figure}

\vspace{0.5cm} % ← espacio después de la figura

\end{document}



\section{Resultados} %Mediante capturas de pantalla y una breve descripción seguida de la captura se presentan los resultados finales de su aplicación.
\subsection*{Archivos}
\begin{figure} [H]
    \centering
    \includegraphics[width=0.6\linewidth]{PRACTICA-111213/Reporte/Borradores/Archivos1.png}
    \caption{Creando un archivo de ejemplo y escribiendo texto en él.}
\end{figure}
\begin{figure} [H]
    \centering
    \includegraphics[width=0.6\linewidth]{PRACTICA-111213/Reporte/Borradores/Archivos2.png}
    \caption{Leyendo el archivo creado.}
\end{figure}
\begin{figure} [H]
    \centering
    \includegraphics[width=0.6\linewidth]{PRACTICA-111213/Reporte/Borradores/Archivos3.png}
    \caption{Sobrescribiendo el archivo creado y leyéndolo nuevamente para corroborar el resultado.}
\end{figure}
\subsection*{Hilos}
Implementación donde se busca demostrar los procesos independientes de los hilos al ejecutar el programa.
\begin{figure} [H]
    \centering
    \includegraphics[width=0.45\linewidth]{PRACTICA-111213/Reporte/Borradores/Hilos1.png}
    \caption{Salida del ejemplo 1.}
\end{figure}
Implementación donde se ejemplifica la ejecución de hilos de manera secuencial (uno espera al otro).
\begin{figure} [H]
    \centering
    \includegraphics[width=0.45\linewidth]{PRACTICA-111213/Reporte/Borradores/Hilos2.png}
    \caption{Salida del ejemplo 2.}
\end{figure}
Ejemplo básico de comunicación entre hilos.
\begin{figure} [H]
    \centering
    \includegraphics[width=0.45\linewidth]{PRACTICA-111213/Reporte/Borradores/Hilos3.png}
    \caption{Salida del ejemplo 3.}
\end{figure}
Implementación donde se aprecia la repartición de tareas hacia cada hilo.
\begin{figure} [H]
    \centering
    \includegraphics[width=0.6\linewidth]{PRACTICA-111213/Reporte/Borradores/Hilos4.png}
    \caption{Salida del ejemplo 4.}
\end{figure}
Ejemplo de comunicación y paso de mensajes entre los hilos y su interacción con el método principal.
\begin{figure} [H]
    \centering
    \includegraphics[width=0.7\linewidth]{PRACTICA-111213/Reporte/Borradores/Hilos5.png}
    \caption{Salida del ejemplo 5.}
\end{figure}
Implementación de resolución del problema del ejemplo 4 de manera secuencial, con el objetivo de comparar tiempos de ejecución y consumo de recursos.
\begin{figure} [H]
    \centering
    \includegraphics[width=0.45\linewidth]{PRACTICA-111213/Reporte/Borradores/Hilos6.png}
    \caption{Salida del ejemplo 6.}
\end{figure}
\subsection*{Patrones de diseño}
\begin{figure} [H]
    \centering
    \includegraphics[width=0.7\linewidth]{PRACTICA-111213/Reporte/Borradores/Patrones1.png}
    \caption{Ejecución del programa con pokemones de prueba.}
\end{figure}

\section{Conclusiones} %Se presenta un análisis de los resultados obtenidos, donde se destaca la importancia de la aplicación de los conceptos teóricos para resolver el problema. No es describir si les gustó la actividad o no. No es decir qué se obtuvo de la práctica. No es describir lo que fue difícil.
La identificación de las clases y métodos de entrada/salida de datos evidenció la correcta implementación del manejo de archivos, permitiendo al sistema la persistencia de la información.\\
El análisis de los ejemplos de concurrencia facilitó la comprensión de la gestión de hilos, mostrando cómo se estructuran las tareas para un procesamiento eficiente y simultáneo.\\
Además, la interpretación de los patrones de diseño presentes en la arquitectura del código mostró la importancia de utilizar soluciones ya establecidas para garantizar un diseño extensible y sostenible. La elaboración de los Diagramas UML complementó el proceso, reforzando la habilidad de modelar y documentar sistemas de software complejos y confirmando la comprensión de los conceptos requeridos en esta práctica.

\printbibliography %Agregar las referencias en bib y mandarlas llamar en cada sección. Ejemplo de cita: ~\cite{10.5555/576122}
\end{document}