\documentclass[letterpaper,12pt,oneside]{article}

\usepackage[top=1in, left=1.25in, right=1.25in, bottom=1in]{geometry}
\usepackage[T1]{fontenc}
\usepackage[utf8]{inputenc}
\usepackage[spanish,es-nodecimaldot,es-tabla]{babel}
\usepackage{caption, subcaption}
\usepackage{graphicx}
\usepackage{array}
\usepackage{tikz}
\usepackage{imakeidx}
\usepackage[style=numeric]{biblatex}
\usepackage{csquotes}
\usepackage{placeins}
\usepackage{float}
\usepackage{setspace}

\addbibresource{./bib/protocolo.bib}
\graphicspath{./figs/}

\begin{document}
\begin{titlepage}
    \centering
    \includegraphics[width=1\textwidth]{Carátula reportes POO.pdf}
\end{titlepage}

\tableofcontents
\clearpage

\section{Introducción} %Este apartado debe abordar los siguientes puntos:

\begin{itemize}
    \item \textbf{Planteamiento del problema:}\\ %Se hace una descripción del problema a resolver.
    Desarrollar un programa que registre distintos tipos de vehículos y calcule el costo de su servicio según sus características, que interactúe con el usuario a través de la consola y valide los datos capturados, para ello se requiere incluir manejo de excepciones para controlar los errores generados por las entradas inválidas, así como otros conceptos teóricos. Además de documentar el funcionamiento mediante diagramas UML.
    
    \item \textbf{Motivación:}\\ %Se describe por qué es necesario dar solución al problema.
    La migración hacia el ecosistema de Dart y Flutter es un paso para acercarnos a herramientas de desarrollo modernas y orientadas a entornos multiplataforma. Comprender cómo los conceptos de POO se aplican en un lenguaje diferente amplia la capacidad de adaptación a nuevos lenguajes y entornos. Además, el correcto manejo de excepciones es indispensable en aplicaciones que dependen de la interacción con el usuario. De igual manera, representar el funcionamiento de una aplicación mediante diagramas UML fomenta habilidades de documentación más formales.
    
    \item \textbf{Objetivos:}\\ %Lo que se espera obtener al darle solución al problema.
    Interpretar el uso y el propósito específico del manejo de excepciones dentro de un programa, identificar las implementaciones de conceptos fundamentales de la Programación Orientada a Objetos en Dart, así como desarrollar diagramas UML (estático y dinámico) que faciliten la comprensión de la estructura del programa y como parte de la documentación formal del mismo, culminando así el proceso de migración de Java al ecosistema de Dart y Flutter.
\end{itemize}

\section{Marco Teórico} %Son los conceptos aplicados, principalmente los vistos en clase, con los que se va a dar solución al problema, explicados de forma concreta, ya sea con su interpretación personal o bien con referencias bibliográficas sólidas.

\subsection{Abstracción}
La abstracción consiste en identificar las características esenciales de un objeto y omitir detalles que no son relevantes para el problema que se desea resolver. Esto nos permite diseñar clases que representan conceptos generales y establecer una base común para clases más específicas.

\subsection{Encapsulamiento}
El encapsulamiento es el mecanismo que protege los datos internos de un objeto, restringiendo su acceso mediante modificadores de visibilidad. En Dart, el prefijo ``\_'' indica que un atributo es privado a nivel de archivo. El uso de getters y setters permite validar datos y mantener la integridad de los valores asignados.

\subsection{Herencia}
La herencia permite crear nuevas clases basadas en una clase ya existente, reutilizando sus atributos y métodos. Gracias a este principio, las clases derivadas pueden extender o redefinir comportamientos. En este proyecto, \texttt{Auto}, \texttt{Moto} y \texttt{Camion} heredan de la clase abstracta \texttt{Vehiculo}.

\subsection{Polimorfismo}
El polimorfismo permite que distintas clases implementen métodos con el mismo nombre pero con comportamientos diferentes. Esto facilita procesar objetos heterogéneos de manera uniforme. En esta práctica, cada tipo de vehículo implementa su propia versión de los métodos \texttt{calcularServicio()} y \texttt{generarReporteServicio()}.

\subsection{Interfaces}
Una interfaz define un conjunto de métodos que una clase debe implementar sin especificar su comportamiento. En Dart, las interfaces ayudan a establecer contratos claros dentro del diseño del software. La interfaz \texttt{ServicioTaller} asegura que todos los vehículos posean los métodos necesarios para calcular costos y generar reportes.

\subsection{Manejo de Excepciones}
El manejo de excepciones permite identificar y gestionar errores durante la ejecución del programa. Dart ofrece mecanismos como \texttt{throw}, \texttt{try}, \texttt{catch} y \texttt{finally} para controlar situaciones inesperadas. En esta práctica, se utilizan para validar datos del usuario y evitar fallos en el flujo del programa.

\subsection{UML (Lenguaje de Modelado Unificado)}
El Lenguaje de Modelado Unificado (UML) es un estándar utilizado para representar visualmente la estructura y el comportamiento de un sistema. Los diagramas de clases muestran relaciones jerárquicas y atributos, mientras que los diagramas de secuencia representan la interacción dinámica entre objetos. Su uso facilita la documentación y comprensión del diseño del software. 


\section{Desarrollo} %Es la descripción de la implementación realizada en el lenguaje de programación, así como las pruebas realizadas para obtener los resultados. No se debe mostrar código, solo describir funciones o la aplicación de los conceptos teóricos. Las pruebas es describir las entradas ingresadas y las salidas obtenidas.

\subsection*{Clase 1}

\subsection*{Clase 2}

\subsection*{Clase 3}

\subsection*{Clase 4}

\subsection*{Diagrama de clases (UML estático)}
%Se presenta un diagrama de clases que modela la aplicacion de forma estática
\begin{figure} [H]
    \centering
    \includegraphics[width=1.05\linewidth]{PRACTICA-09y10/Reporte/Borradores/DiagramaClases.png}
    \caption{UML:Diagrama de clases.}
    \end{figure}
    
\subsection*{Diagrama de secuencia (UML dinámico)}
\begin{figure}[H]
    \centering
    \includegraphics[width=0.9\linewidth]{PRACTICA-09y10/Reporte/Borradores/Diagrama dinamico.png}
    \caption{UML: Diagrama Dinámico.}
    \label{fig:diagrama-dinamico}
\end{figure}




\section{Resultados} %Mediante capturas de pantalla y una breve descripción seguida de la captura se presentan los resultados finales de su aplicación.
\begin{figure} [H]
    \centering
    \includegraphics[width=0.45\linewidth]{PRACTICA-09y10/Reporte/Borradores/AgregarAuto.png}
    \caption{Agregando un auto al sistema de taller mecánico.}
\end{figure}
\begin{figure} [H]
    \centering
    \includegraphics[width=0.45\linewidth]{PRACTICA-09y10/Reporte/Borradores/AgregarMoto.png}
    \caption{Agregando una moto al sistema de taller mecánico.}
\end{figure}
\begin{figure} [H]
    \centering
    \includegraphics[width=0.45\linewidth]{PRACTICA-09y10/Reporte/Borradores/AgregarCamion.png}
    \caption{Agregando un camión al sistema de taller mecánico.}
\end{figure}
\begin{figure} [H]
    \centering
    \includegraphics[width=0.7\linewidth]{PRACTICA-09y10/Reporte/Borradores/Flotilla.png}
    \caption{Imprimiendo un resumen de la flotilla.}
\end{figure}
\begin{figure} [H]
    \centering
    \includegraphics[width=0.45\linewidth]{PRACTICA-09y10/Reporte/Borradores/Reporte.png}
    \caption{Imprimiendo reporte detallado de la flotilla.}
\end{figure}
\begin{figure} [H]
    \centering
    \includegraphics[width=0.5\linewidth]{PRACTICA-09y10/Reporte/Borradores/Saliendo.png}
    \caption{Saliendo del sistema de taller mecánico.}
\end{figure}

\section{Conclusiones} %Se presenta un análisis de los resultados obtenidos, donde se destaca la importancia de la aplicación de los conceptos teóricos para resolver el problema. No es describir si les gustó la actividad o no. No es decir qué se obtuvo de la práctica. No es describir lo que fue difícil.

Esta práctica supuso la migración de Java al ecosistema de Dart y Flutter, lo que permitió identificar similitudes y diferencias clave en la aplicación de los principios de la Programación Orientada a Objetos entre ambos lenguajes. El programa emplea principios fundamentales como las clases abstractas, interfaces, herencia y polimorfismo, lo cual permitió establecer una conexión con los conceptos teóricos abordados en clases, particularmente para el caso de las excepciones, se logró evidenciar su importancia en un programa para garantizar un flujo correcto de operaciones durante la ejecución.\\
Además, como parte de la documentación de la práctica, los diagramas UML permitieron visualizar la estructura general del programa con el diagrama estático de clases y el flujo de interacción entre el programa y el usuario con el diagrama dinámico de secuencia de una manera práctica e intuitiva. Esto demostró la utilidad del UML para la comprensión del diseño de software y reafirmó la importancia de mantener una documentación formal en los proyectos.\\
Con esta práctica se pudo comprobar que los principios esenciales de la Programación Orientada a Objetos se mantienen constantes y aplicables, independientemente del entorno utilizado.
\printbibliography %Agregar las referencias en bib y mandarlas llamar en cada sección. Ejemplo de cita: ~\cite{10.5555/576122}
\end{document}
